\documentclass[a4paper,10pt]{article}
\title{}
\author{}
\usepackage{graphicx} 
\usepackage{amsthm}
\usepackage{amsmath}
\usepackage{amsfonts} 
\usepackage{amsopn}
\usepackage{asymptote}
\parskip 7.2pt
%% Missing trig. and hyp. operators
\newcommand{\cosec}{\ensuremath{\mathrm{cosec}}}
\newcommand{\arccosec}{\ensuremath{\mathrm{arccosec}}}
\newcommand{\arcsec}{\ensuremath{\mathrm{arcsec}}}
\newcommand{\arccot}{\ensuremath{\mathrm{arccot}}}
\newcommand{\arcsinh}{\ensuremath{\mathrm{arcsinh}}}
\newcommand{\arccosh}{\ensuremath{\mathrm{arccosh}}}
\newcommand{\arctanh}{\ensuremath{\mathrm{arctanh}}}
\newcommand{\arccoth}{\ensuremath{\mathrm{arccoth}}}
\DeclareMathOperator{\argmax}{arg\,max}
\newtheorem{theorem}{Theorem}[section]
\newtheorem{definition}{Definition}[section]

\begin{document}
\maketitle
\section{introduction}

We consider closed inner orbits connecting edges of acute angled triangles in the
hyperbolic plane and bound from above the ratio between the Fangano orbit -the
shortest orbit connecting the edges of a triangle and the the triangle's
circumference by $\frac{1}{2}$ - a ratio that is only achieved in the equilateral
Euclidean triangle and is naturally the solution to the Euclidean version of the
same problem [1].
In fact we show that the shortest orbit connecting all three edges coincides with
the shortest inner billiard and with the more constructive inner triangle 
connecting the intersections of the altitudes and the edges.

Since the domain of our problem is acute angled hyperbolic triangles we
limit ourselves to the three dimensional closed* polytop:

\begin{center}
   $\mathcal{P} = \left\{\alpha,\beta,\gamma| \alpha + \beta + \gamma <= \pi, 0 <= \alpha,\beta,\gamma <= \frac{\pi}{2}\right\}$
\end{center}

We use section 1 to prove the three closed orbits indeed coincide for
every triangle in our domain of interest, section 2 to express the length of
the orbit and the circumference as a function of the angles, section 3
to prove the ratio between the two is bounded by $\frac{1}{2}$ on the
interior of the polytop and section 4 to show the same holds on the
boundary.


*We allow ourselves to consider the closed polytop even though formally
$\alpha + \beta + \gamma < \pi$ for triangle on the hyperboloc plane and 
equality is never achived. We do so since all of our expressions can be shown
to be continuous with respect to the Euclidean case in the sense that their 
values tend to the natural Euclidean version of the expression as the
sum of the angles approaches $\pi$. We formulate and prove the above in
appendix A.

\begin{figure}
\centering
\begin{asy}
size (3cm);
import math;
import hyperbolic_geometry;
import fontsize;

unitsize(3cm);
hyperbolic_point A = hyperbolic_point(2.7, 0);
hyperbolic_point B = hyperbolic_point(2.6, 100);
hyperbolic_point C = hyperbolic_point(1.2, 250);

draw(hyperbolic_segment(A,B));
draw(hyperbolic_segment(B,C));
draw(hyperbolic_segment(C,A));


pair EUA = A.get_euclidean();
pair EUB = B.get_euclidean();
pair EUC = C.get_euclidean();

pen p = fontsize(1);
label("$A$",EUA,E,p);
label("$\alpha$", (EUA.x - 0.2, EUA.y) , W, p);
label("$a$", (EUC + EUB)/2, W, p);
draw(arc(EUA, 0.3, 150, 210, true));

label("$B$",(EUB.x, EUB.y + 0.1), NW, p);
label("$\beta$", (EUB.x + 0.12, EUB.y - 0.35) , W, p);
label("$b$", (EUC + EUA)/2, SW, p);
draw(arc(EUB, 0.4, 260, 310, true));

label("$C$",(EUC.x, EUC.y - 0.05), SW, p);
label("$\gamma$", (EUC.x + 0.1, EUC.y + 0.15) , W, p);
label("$c$", (EUB + EUA)/2, SE, p);
draw(arc(EUC, 0.3, 30, 100, true));

path frame = (-1,-1)--(-1,1)--(1, 1)--(1,-1)--cycle;
clip(frame);
\end{asy}
\caption{Notation for vertices, adges and angles}
\label{fig-notation}
\end{figure}


\section{Notation}

All through the text, points in the hyperbolic plane are marked
with Latin capitals. When using a letter, say $A$ in a specific model
it may refer either to the point in the hyperbolc plane or to the point
representing it in the model, which is used can be deduced by context.

For two different points $A$ and $B$, $AB$ marks the segment connecting
point $A$ and point $B$, or the model specific representation of the segment
- which need not be an Euclidean segment.
It is important to note that in Euclidean embeded models the Euclidean segment
and hyperbolic segment do not necessarily coincide, in fact they usually do not,
in those cases $AC$ refers to the hyperbolic segment and not the Euclidean segment.

In some cases $AB$ may refer to the geodesic on which segment $AB$ is part of.
In those cases it is specifically noted by refering to it as 'the geodesic $AB$'.

Triangle vertices are marked with the capital Latins $A$, $B$ and $C$.
Hyperbolic lengths of edges are marked $a$, $b$ and $c$, each refering to 
the length of the edge opposite
to the matching vertex, $a$ for example refers to $d\left(B, C\right)$ 
Angles $\alpha$, $\beta$ and $\gamma$ refer to $\alpha = \angle BAC$, 
$\alpha = \angle CBA$ and $\alpha = \angle ACB$ respectfully
(see Figure \ref{fig-notation}).

The terms Geodesics, Hyperbolic distances, segment lenghs angles and 
triangles are all defined and explained in section \ref{sec-hyperbolicgeom}.


\label{sec-hyperbolicgeom}
\section{Hyperbolic Geometry}
\subsection{Short introduction to Hyperbolic Geometry}
\subsection{Hyperbolic Trigonometry}

\label{cosine1}
\begin{theorem}
First Hyperbolic law of cosines
\end{theorem}

\label{cosine2}
\begin{theorem}
First Hyperbolic law of cosines
\end{theorem}

\label{hyperbolicheightexistance}
\begin{definition}

\end{definition}


\label{hyperbolicheights2}
\begin{theorem}
The heights of an acute angled hyperbolic triangle are concurent
\end{theorem}

\label{hyperbolicheights2}
\begin{theorem}
In an acute angled triangle $ABC$, the height leaving $A$ crosses $BC$.
\end{theorem}

\subsection{The half plane model}

Note: in the following text, half plane points are sometimes parameterized by the
value of their x coordinate and their suqare norm, for example:
\[
P = \left(x, \sqrt{r - x^2}\right)\\
\]
In which cases we always implicitly assume $x$ is in $\mathbb{R}$ and $s$, the square norm, is in $\mathbb{R_+}$.

\label{half-plane-concurrent}
\begin{theorem}

Let
\[
P_i = \left(x_i, \sqrt{s_i - x_i^2}\right), i\in\left\{0,1,2\right\}
\]

Be three distinct points on the same geodesic, then:

\[
  \left(x_2 - x_1\right)\left(s_3 - s_2\right) = \left(x_3 - x_2\right)\left(s_2 - s_1\right)
\]
\end{theorem}
\begin{proof}
...
\end{proof}

\label{half-plane-concurrent}
\begin{theorem}

Let
\[
P_i = \left(x_i, \sqrt{s_i - x_i^2}\right), i\in\left\{1,2,3,4\right\}
\]



Be four distinct points such that the geodesics $P_1P_2$ and $P_3P_4$ intersect the y-axis at
points $(0,\sqrt{u})$ and $(0,\sqrt{v})$ (need not be unique, geodesics may coincide with the y-axis)
then the two geodescis intersect each other and are perpendicular if and only if

\[
2\left(x_1 - x_2\right)\left(x_3 - x_4\right)\left(u + v\right) + \left(s_1 - s_2\right)\left(s_3 - s_4\right) = 0
\]

\end{theorem}

\begin{proof}
...
\end{proof}

Note: A condition for the fact that the two geodescis meet (not necesseraily being perpendiular) 
is not as crisp and is not needed in the text and for that reason not detailed.

\section{Closed orbits of acute angled hyperbolic triangles}

We start by showing a Hyperbolic version of the three relatively trivial claims:

(I) There is a unique inner billiard connecting all three edges.

(II) The shortest inner billiard (Fangano orbit) is the above unique billiard.

(III) The shortest closed inner orbit passing throgh all edges is a billiard

And will continue on to show a less trivial fact:

(IV) The orthic triangle (The triangle connecting the base points of the 
three altitudes is an inner billiard.

\begin{proof}

Let $D,E,F$ be the unique points on $BC,CA,AB$ such that $AD$ is perpendicular to $BC$,
$BE$ is perpendicular to $CA$ and $FC$ is perpendicular to $AB$
(Exist and unique due to \ref{hyperbolicheights2}).

$A$ to $BC$, $E$ be the 
projection of point $B$ to $AC$, $F$ be the projection of point $C$ to $AB$.
Let $Q$ be the reflection of $D$ with respect to $AB$. Let $J$ be

Let $G$ be the intersection of the $y$ axis and $EQ$.

Claim: $G$ coincides with $F$.

Proof of this claim prooves $\angle EDA = \angle FDB$, and by symmetry
prooves the mirror property of all vertices of the orbit.

Proof of Claim:

We fomulate the proof in the upper half plane model, relying of the well 
known fact that the three altitudes of a triangle meet at one point. In an
effort to simplify our calculations, we parameterize our points by their $x$ 
coordinate and their squared distance from the origin:

\begin{align*}
A&=&\left(0,\sqrt{a}\right)
B&=&\left(0,\sqrt{b}\right)\\
C&=&\left(c, \sqrt{r - c^2}\right)
D&=&\left(x, \sqrt{s - x^2}\right)\\
Q&=&\left(-x, \sqrt{s - x^2}\right)
E&=&\left(y, \sqrt{t - y^2}\right)\\
F&=&\left(0,\sqrt{f}\right)
G&=&\left(0,\sqrt{g}\right)
\end{align*}


\begin{figure}
\centering
\begin{asy}
import math;
import graph;
size (6cm);

pair hlinecenter(pair a, pair b)
{
  return (a*conj(a) - b*conj(b))/2/(xpart(a)-xpart(b));
}


import math;
import graph;
unitsize(0.5cm);
pair O = (0,0);
pair A = (0,3);
pair B = (0,9);
pair C = (6,2);

pair cAC = hlinecenter(A,C);
pair cBC = hlinecenter(B,C);
draw(arc(cAC, length(A-cAC), degrees(A-cAC), degrees(C-cAC), false));
draw(arc(cBC, length(B-cBC), degrees(B-cBC), degrees(C-cBC), false));
draw(arc(O, length(C), degrees(C), 90, true));

draw((-10,0)--(10,0));
draw((0,0)--(0,10));
\end{asy}
\caption{Triangle modeled on the Poincare Upper half plane.}
\label{fig:mirror}
\end{figure}



Let us show that $f = g$.

\noindent
We will rely on two half plane identities:

Let us now proove $f=g$.
We will start by expressing $x$ and $s$ by $a$,$b$,$c$ and $r$, we know that:

\begin{align*}
&b = \frac{cs - xr}{c - x} \\
&\frac{s - a}{2x} \frac{r - b}{2c} = -\frac{a+b}{2}
\end{align*}

Solving for $x$, $s$:
\begin{align*}
x &= \frac{c\left(r - a\right)\left(b - a\right)}{\left(r - a\right)^2 + 2c^2\left(a + b\right)}\\
s &= \frac{b\left(r - a\right)^2 + 2c^2\left(a + b\right)a}{\left(r - a\right)^2 + 2c^2\left(a + b\right)}
\end{align*}

And by symmetry:
\begin{align*}
y &= \frac{c\left(r - b\right)\left(a - b\right)}{\left(r - b\right)^2 + 2c^2\left(b + a\right)}\\
t &= \frac{a\left(r - b\right)^2 + 2c^2\left(b + a\right)b}{\left(r - b\right)^2 + 2c^2\left(b + a\right)}
\end{align*}

Now Let us find $F$ as the intersection of $QE$ and the $y$ axis:

\begin{align*}
g &= \frac{ys - (-x)t}{y - (-x)} = \frac{ys + xt}{y + x)}\\
   &= \frac{ \frac{c\left(r - b\right)\left(a - b\right)}{\left(r - b\right)^2 + 2c^2\left(b + a\right)} \frac{b\left(r - a\right)^2 + 2c^2\left(a + b\right)a}{\left(r - a\right)^2 + 2c^2\left(a + b\right)} + \frac{c\left(r - a\right)\left(b - a\right)}{\left(r - a\right)^2 + 2c^2\left(a + b\right)} \frac{a\left(r - b\right)^2 + 2c^2\left(b + a\right)b}{\left(r - b\right)^2 + 2c^2\left(b + a\right)} }{ \frac{c\left(r - b\right)\left(a - b\right)}{\left(r - b\right)^2 + 2c^2\left(b + a\right)} + \frac{c\left(r - a\right)\left(b - a\right)}{\left(r - a\right)^2 + 2c^2\left(a + b\right)} }\\
  &= \frac{c\left(r - b\right)\left(a - b\right) \left(b\left(r - a\right)^2 + 2c^2\left(a + b\right)a\right) + c\left(r - a\right)\left(b - a\right) \left(a\left(r - b\right)^2 + 2c^2\left(b + a\right)b\right) }{c\left(r - b\right)\left(a - b\right)\left(\left(r - a\right)^2 + 2c^2\left(a + b\right)\right) + c\left(r - a\right)\left(b - a\right)\left(\left(r - b\right)^2 + 2c^2\left(b + a\right)\right) }\\
  &= \frac{\left(r - b\right) \left(b\left(r - a\right)^2 + 2c^2\left(a + b\right)a\right) - \left(r - a\right) \left(a\left(r - b\right)^2 + 2c^2\left(b + a\right)b\right) }{\left(r - b\right)\left(\left(r - a\right)^2 + 2c^2\left(a + b\right)\right) - \left(r - a\right)\left(\left(r - b\right)^2 + 2c^2\left(b + a\right)\right) }\\
  &= \frac{2c^2\left(a + b\right)\left(a\left(r - b\right) - b\left(r - a\right)\right) + \left(r - a\right)\left(r - b\right)\left( b\left(r - a\right) - a\left(r - b\right)\right)}{2c^2\left(a + b\right)\left(r - b - r + a\right) + \left(r - a\right)\left(r - b\right)\left(\left(r - a\right) - \left(r - b\right)\right)}\\
  &= \frac{2c^2\left(a + b\right)r\left(a - b\right) + \left(r - a\right)\left(r - b\right)r\left(b - a\right)}{2c^2\left(a + b\right)\left(a - b\right) + \left(r - a\right)\left(r - b\right)\left(b - a\right)}\\
&= r
\end {align*}
\end{proof}

\section{Equilateral Triangles}

The family of hyperbolic equilateral triangles is a single parameter 
family. The parameter can be the edge length or the angle size, we will
choose to parameterize the family by the angle, $\alpha$, that can be
any where in the half open interval $\left[0, \frac{\pi}{3}\right)$.

Claim: The ratio between the Fangano orbit length and the perimiter
of the triangle is monotone with respect to $\alpha$.

Corollary: The ratio between the Fangano orbit length and the perimiter
is smaller than $2$.

We will show the two following identities:
\begin{align}
\cosh{\frac{f}{3}}&=\cos{\alpha}+\frac{1}{2}\label{EquilateralFanganoOrbitLength}\\
\cosh{\frac{p}{3}}&=\frac{\cos{\alpha}}{1-\cos{\alpha}}\label{EquilateralPerimiter}\\
\end{align}
Where $f$ is the Fangano orbit length and $p$ is the perimiter.

Proof:

\begin{figure}
\centering
\begin{asy}
size (3cm);
import math;
import hyperbolic_geometry;
unitsize(3cm);
hyperbolic_point O = hyperbolic_point(0, 0);
hyperbolic_point A = hyperbolic_point(2.7, 0);
hyperbolic_point B = hyperbolic_point(2.7, 120);
hyperbolic_point C = hyperbolic_point(2.7, 240);


hyperbolic_point D = basepoint(hyperbolic_line(B,C), A);
hyperbolic_point hE = basepoint(hyperbolic_line(C,A), B);
hyperbolic_point F = basepoint(hyperbolic_line(A,B), C);
hyperbolic_point J = intersection(hyperbolic_line(hE,F), hyperbolic_line(A,D));


draw(unitcircle);
draw(hyperbolic_segment(A,D));
draw(hyperbolic_segment(A,B));
draw(hyperbolic_segment(A,C));
draw(hyperbolic_segment(B,C));
draw(hyperbolic_segment(F,hE));
draw(hyperbolic_segment(D,hE));
draw(hyperbolic_segment(F,D));
pen p = fontsize(1);
label("$J$",J.get_euclidean(),NE,p);
label("$A$",A.get_euclidean(),E,p);
label("$B$",B.get_euclidean(),NW,p);
label("$C$",C.get_euclidean(),SW,p);
label("$D$",D.get_euclidean(),W,p);
label("$E$",hE.get_euclidean(),SE,p);
label("$F$",F.get_euclidean(),NE,p);
path frame = (-1,-1)--(-1,1)--(1, 1)--(1,-1)--cycle;
clip(frame);
\end{asy}
\caption{Equilaterlal Triangle modeled on the Poincare Disc.}
\label{fig:equi-orbit}
\end{figure}


\begin{align*}
\sinh{\frac{p}{6}}&=\sinh{BD}=\sinh{AB}\sin{\frac{\alpha}{2}}=\sinh{\frac{p}{3}}\sin{\frac{\alpha}{2}}\\
&\Downarrow\\
\sinh{\frac{p}{6}} &= \sinh{2\frac{p}{6}}\sin{\frac{\alpha}{2}}\\
&\Downarrow\\
\sinh{\frac{p}{6}} &= 2\sinh{\frac{p}{6}}\cosh{\frac{p}{6}}\sin{\frac{\alpha}{2}}\\
&\Downarrow\\
\cosh{\frac{p}{6}} &= \frac{1}{2\sin{\frac{\alpha}{2}}}\\
&\Downarrow\\
\cosh{\frac{p}{3}} &= \frac{2}{4\sin^2{\frac{\alpha}{2}}} - 1 = \frac{1}{2\sin^2{\frac{\alpha}{2}}} - 1 \\
&= \frac{1}{1-\cos{\alpha}} - 1 = \frac{\cos{\alpha}}{1-\cos{\alpha}} \\
\end{align*}
Which proves \eqref{EquilateralPerimiter}


\begin{align*}
\cosh{\frac{f}{3}}&=\cosh{2\frac{f}{6}}\\
&=2\sinh^2{\frac{f}{6}}+1=2\sinh^2{FJ}+1\\
&=2\sinh^2{FA}\sin^2{\frac{\alpha}{2}}+1 = \left(\cosh{2FA}-1\right)\frac{1-\cos{\alpha}}{2}+1\\
&=\left(\cosh{\frac{p}{3}}-1\right)\frac{1-\cos{\alpha}}{2}+1=\left(\frac{\cos{\alpha}}{1-\cos{\alpha}}-1\right)\frac{1-\cos{\alpha}}{2}+1\\
&=\frac{1}{2}\left(\cos{\alpha}-1+\cos{\alpha}\right)+1=\cos{\alpha}+\frac{1}{2}
\end{align*}
Which proves \eqref{EquilateralFanganoOrbitLength}.

For $\alpha=0$, the perimiter is infinite while the orbit is finite, so $\frac{f}{a}$ is zero.
Inorder to show that the ratio $\frac{f}{a}$ is monotone with respect to $\alpha$ it
is enough to show that:

\[
\frac{f_\alpha}{a_\alpha} > \frac{f}{a}
\]

Where $f_\alpha$ and $a_\alpha$ are the deriviatives of $f$ and $a$ with respect to $\alpha$.

Let $t=\cos{\alpha}$ and $s=\sin{\alpha}$:

\begin{align*}
\frac{f_\alpha}{a_\alpha} &= \frac{  \frac{s}{\sqrt{  \left(t+\frac{1}{2}\right)^2   - 1} }}
                                {  \frac{s}{\sqrt{  \left(  \frac{t}{1-t}  \right)^2   - 1} }}
= \sqrt{\frac{\left(  \frac{t}{1-t}  \right)^2   - 1}{  \left(t+\frac{1}{2}\right)^2   - 1 }}
= \sqrt{\frac{\left(  \frac{1}{1-t}  \right)\left(  \frac{2t-1}{1-t}  \right)}
              {  \left(t-\frac{1}{2}\right)\left(t+\frac{3}{2}\right) }}\\
&= \sqrt{\frac{2\left( t-\frac{1}{2}\right)}
              {  \left(1 - t\right)^2\left(t-\frac{1}{2}\right)\left(t+\frac{3}{2}\right) }}
= \sqrt{\frac{2}
              {  \left(1 - t\right)^2\left(t+\frac{3}{2}\right) }}\\
\end{align*}


\section{The general case - Expressing lengths as functions of angles}

In this section we will show the two following identities:

\begin{align}
&\sinh{\frac{f}{2}}=\Phi \label{FanganoOrbitLength}\\
&\sinh{\frac{s}{2}}=\frac{\Phi}{4\sin{\frac{\alpha}{2}}\sin{\frac{\beta}{2}}\sin{\frac{\gamma}{2}}}\label{Perimiter}\\
&\tanh{r}=\frac{\Phi}{4\cos{\frac{\alpha}{2}}\cos{\frac{\beta}{2}}\cos{\frac{\gamma}{2}}}\label{Inscribed}
\end{align}

Where $f$ is the length of the Fangano orbit, $s$ the perimiter, $r$ 
the radius of the inscribed (hyperbolic) circle and:
\[
\Phi^2=2\cos{\alpha}\cos{\beta}\cos{\gamma}+\cos^2{\alpha}+\cos^2{\beta}+\cos^2{\gamma}-1
\]
\subsection{Proof}

We consider the acute angled hyperbolic triangle $\triangle ABC$. Let
 $a$, $b$ and $c$ be the lengths of $BC$, $CA$, $AB$ respectfully and
$\alpha$, $\beta$ and $\gamma$ the angles $\angle BAC$, $\angle ABC$,
$\angle BCA$.  Let $D$ be the projection of point $A$ to $BC$, and $d$ 
be the length of $AD$. Let $P$ be the reflection of $D$ with respect
to $AC$ and $Q$ the reflection of $D$ with respect to $AB$.
(The construction is demonstrated in Figure \ref{fig:orbit}).

\begin{figure}
\centering
\begin{asy}
size (6cm);
import math;
import hyperbolic_geometry;
unitsize(3cm);
hyperbolic_point A = hyperbolic_point(0,0);
hyperbolic_point B = hyperbolic_point(2.7, 63);
hyperbolic_point C = hyperbolic_point(1.7, 13);
hyperbolic_point D = basepoint(hyperbolic_line(B,C), A);
hyperbolic_point E = basepoint(hyperbolic_line(C,A), B);
hyperbolic_point F = basepoint(hyperbolic_line(A,B), C);
hyperbolic_point P = mirror(hyperbolic_line(A,B), D);
hyperbolic_point Q = mirror(hyperbolic_line(A,C), D);
draw(unitcircle);
draw(hyperbolic_segment(A,B));
draw(hyperbolic_segment(A,C));
draw(hyperbolic_segment(B,C));
draw(hyperbolic_segment(A,D));
draw(hyperbolic_segment(P,Q));
draw(hyperbolic_segment(D,E));
draw(hyperbolic_segment(F,D));
draw(hyperbolic_segment(A,Q));
draw(hyperbolic_segment(A,P));
pen p = fontsize(1);
label("$A$",A.get_euclidean(),SW,p);
label("$B$",B.get_euclidean(),NE,p);
label("$C$",C.get_euclidean(),NE,p);
label("$D$",D.get_euclidean(),NE,p);
label("$E$",E.get_euclidean(),SW,p);
label("$F$",F.get_euclidean(),W,p);
label("$P$",P.get_euclidean(),NE,p);
label("$Q$",Q.get_euclidean(),NE,p);
path frame = (-0.3,-0.3)--(-0.3,1.1)--(1.1, 1.1)--(1.1,-0.3)--cycle;
clip(frame);
\end{asy}
\caption{Triangle modeled on the Poincare Disc.}
\label{fig:orbit}
\end{figure}

First, since $DEF$ is an inner billiard, and since $P$ and $Q$ are
both reflections of $D$, it easy to see that $P$, $F$, $E$ and $Q$ are
on a common geodesic and that $DF = PF$ and $ED=EQ$. So:
\begin{align*}
2f &= \overline{DF} + \overline{FE} + \overline{ED}\\
     &= \overline{PF} + \overline{FE} + \overline{EQ}\\
     &= \overline{PQ}
\end{align*}

Second, from the same exact properties, $\angle PAF = \angle DAF$ and
$\angle QAE = \angle DAE$, and thus:

\begin{align*}
\angle PAQ &= \angle PAD + \angle DAQ = 2\angle BAD + 2\angle DAC \\
           &= 2\left(\angle BAD + \angle DAC \right) = 2\angle BAC \\
           &= 2\alpha
\end{align*}

Using the hyperbolic law of cosines for $\triangle PAQ$ and right
angle identity for $\triangle ADC$ we get the following identities
(respectfully):

\begin{align}
\cosh{2f} &= \cosh^2{d} - \sinh^2{d}\cos{2\alpha} \label{CosineLawPAQ} \\
\sinh{d} &= \sinh{c}\sin{\beta} \label{tanADC}
\end{align}

Develope \eqref{CosineLawPAQ}:

\begin{align*}
\cosh{2f} &=1+ \sinh^2{d} - \sinh^2{d}\left(\cos^2{\alpha} - \sin^2{\alpha}\right) \\
&=1+ \sinh^2{d}\left(1 - \cos^2{\alpha} + \sin^2{\alpha}\right) \\
&=1+2\sinh^2{d}\sin^2{\alpha} \\
\end{align*}

Substituting for $\cosh{2f}$ and $\sinh^2{d}$ (using \eqref{tanADC}), we get:

\begin{align*}
1 + 2\sinh^2{f} &=1+2\sinh^2{c}\sin^2{\beta}\sin^2{\alpha} \\
&=1+2\left(\frac{\cos{\alpha}\cos{\beta} + \cos{\gamma}}{\sin{\alpha}\sin{\beta}}^2 - 1\right)\sin^2{\beta}\sin^2{\alpha} \\
&=1+2\left(\cos{\alpha}\cos{\beta}\cos{\gamma} + \cos^2{\gamma} + \cos^2{\alpha}\cos^2{\beta} - \sin^2{\beta}\sin^2{\alpha} \right) \\
&=1+2\left(\cos{\alpha}\cos{\beta}\cos{\gamma} + \cos^2{\gamma} + \cos^2{\alpha}\cos^2{\beta} -
\left(1-\cos^2{\alpha}\right)\left(1-\cos^2{\beta}\right)\right)\\
&=1+2\left(2\cos{\alpha}\cos{\beta}\cos{\gamma} + \cos^2{\gamma} + \cos^2{\beta} + \cos^2{\alpha} + \cos^2{\beta} - 1\right) \\
&=1+2\Phi^2 \\
\end{align*}

Which proves \eqref{FanganoOrbitLength}:
\[
\sinh{f}=\Phi
\]

...

\section{Bounding ratios between invariants on the interior}

In this section we will show that ${\frac{f}{s}}$ has no critical points
on the interior of $\mathbf{P}$. We will do so by diving $\mathbf{P}$ to
level sets of the perimiter function $s$, and showing that for a given value of 
$s$, the the above ratio achieves its maximum on the boundary of $\mathbf{P}$ 
or on the set $\alpha=\beta=\gamma$. We will then show that on that set 
the ratio is monotone with resppect to, say, $\alpha$.

\subsection{Proof}

Fixing $s$, by \eqref{Perimiter}, means fixing
\[
\frac{\Phi}{4\cos{\frac{\alpha}{2}}\cos{\frac{\beta}{2}}\cos{\frac{\gamma}{2}}}
\]
Thus fixing its square: 
\[
\frac{\Phi^2}{16\cos^2{\frac{\alpha}{2}}\cos^2{\frac{\beta}{2}}\cos^2{\frac{\gamma}{2}}} = \\
\frac{\Phi^2}{2\left(1 - \cos{\alpha}\right)\left(1 - \cos{\beta}\right)\left(1 - \cos{\gamma}\right)}
\]
We shall mark the above ratio by $\rho$.

By \eqref{FanganoOrbitLength}, ${\frac{f}{s}}$ achieves its maximum (given a fixed s),
when ${\Phi}$ achives its maximum and thus when ${\Phi^2}$ achives its maximum. Our 
maximizing problem is thus reduced to maximizing $\Phi^2$, given a fixed value for
$\frac{\Phi^2}{\rho}$. Using Lagrange multipliers:

\begin{align*}
\bigtriangleup \Phi^2 &= \lambda \bigtriangleup\frac{\Phi^2}{\rho}\\
&\Downarrow\\
\frac{1}{\lambda}\bigtriangleup\left(\Phi^2\right) &= \frac{\bigtriangleup\left(\Phi^2\right)\rho - \bigtriangleup\left(\rho\right)\Phi^2}{\rho^2}\\
&\Downarrow\\
\left(\frac{1}{\lambda}-\frac{1}{\rho}\right)\bigtriangleup\left(\Phi^2\right) &= -\frac{\Phi^2}{\rho^2}\bigtriangleup\left(\rho\right)\\
\end{align*}

In other words, $\bigtriangleup\left(\Phi^2\right)$ and $\bigtriangleup\left(\rho\right)$ are 
linearly dependant, meaning the ratios between the directional deriviatives are respectfully equal:
\begin{align*}
  \frac{\Phi^2_\alpha}{\rho_\alpha} = &\frac{\Phi^2_\beta}{\rho_\beta} = \frac{\Phi^2_\gamma}{\rho_\gamma}\\
  &\Downarrow\\
  \frac{-2\left(\cos{\beta}\cos{\gamma} + \cos{\alpha}\right)\sin{\alpha}}{2\left(1 - \cos{\beta}\right)\left(1 - \cos{\gamma}\right)\sin{\alpha}} = 
  &\frac{-2\left(\cos{\gamma}\cos{\alpha} + \cos{\beta}\right)\sin{\beta}}{2\left(1 - \cos{\gamma}\right)\left(1 - \cos{\alpha}\right)\sin{\beta}} = 
  \frac{-2\left(\cos{\alpha}\cos{\beta} + \cos{\gamma}\right)\sin{\gamma}}{2\left(1 - \cos{\alpha}\right)\left(1 - \cos{\beta}\right)\sin{\gamma}}\\
\end{align*}

And Specifically:
\begin{align*}
\frac{\left(\cos{\beta}\cos{\gamma} + \cos{\alpha}\right)}{\left(1 - \cos{\beta}\right)\left(1 - \cos{\gamma}\right)} &=
\frac{\left(\cos{\gamma}\cos{\alpha} + \cos{\beta}\right)}{\left(1 - \cos{\gamma}\right)\left(1 - \cos{\alpha}\right)}\\
  &\Downarrow\\
\frac{\left(\cos{\beta}\cos{\gamma} + \cos{\alpha}\right)}{\left(1 - \cos{\beta}\right)} &=
\frac{\left(\cos{\gamma}\cos{\alpha} + \cos{\beta}\right)}{\left(1 - \cos{\alpha}\right)}\\
  &\Downarrow\\
\cos{\beta}\cos{\gamma} + \cos{\alpha} - \cos{\alpha}\cos{\beta}\cos{\gamma} - \cos^2{\alpha} &=
\cos{\alpha}\cos{\gamma} + \cos{\beta} - \cos{\alpha}\cos{\beta}\cos{\gamma} - \cos^2{\beta}\\
  &\Downarrow\\
\cos{\beta}\cos{\gamma} + \cos{\alpha} - \cos^2{\alpha} &=
\cos{\alpha}\cos{\gamma} + \cos{\beta} - \cos^2{\beta}\\
  &\Downarrow\\
\cos{\gamma}\left(\cos{\beta}-\cos{\alpha}\right) &- \left(\cos^2{\alpha}-\cos^2{\beta}\right) + \cos{\beta} - \cos{\alpha} = 0\\
  &\Downarrow\\
\left(\cos{\beta} - \cos{\alpha}\right)\left(\cos{\gamma} - \cos{\alpha} - \cos{\beta} + 1\right) &=0
\end{align*}

Similraly we derive two more equations and arrive with:
\begin{align*}
\left(\cos{\beta} - \cos{\alpha}\right)\left(\cos{\gamma} - \cos{\alpha} - \cos{\beta} + 1\right) &=0\\
\left(\cos{\gamma} - \cos{\beta}\right)\left(\cos{\alpha} - \cos{\beta} - \cos{\gamma} + 1\right) &=0\\
\left(\cos{\alpha} - \cos{\gamma}\right)\left(\cos{\beta} - \cos{\gamma} - \cos{\alpha} + 1\right) &=0
\end{align*}

Which has only one internal solution $\alpha=\beta=\gamma$.

Explanation:

If, say, $\alpha=\beta$ Then by the second equation (and the fact the we looking for internal solutions), $\gamma$ must
be equal to $\beta$, so a solution other than $\alpha=\beta=\gamma$ must include three distinct angles, meaning:
\begin{align*}
\left(\cos{\gamma} - \cos{\alpha} - \cos{\beta} + 1\right) &=0\\
\left(\cos{\alpha} - \cos{\beta} - \cos{\gamma} + 1\right) &=0\\
\left(\cos{\beta} - \cos{\gamma} - \cos{\alpha} + 1\right) &=0
\end{align*}

Of which the only solution is $\cos{\alpha}=\cos{\beta}=\cos{\gamma}=1$.
We conclude that any internal critical point must lay on the line $\alpha=\beta=\gamma$.

Let us now look at the single variable function:


\section{Bounding the ratio on the boundary}

Up to symmetry, the boundary is assembled by sets of one of the types: 
$\left\{\alpha=0\right\}$, $\left\{\alpha=\frac{\pi}{2}\right\}$, and
$\left\{\alpha+\beta+\gamma=0\right\}$. The first is trivial since on it the
circumference is infinite and the orbit is finite (since it connects
three internal points), the last reduces to the known Euclidean case,
which leaves us with the case that one of the angles is a right
angle (since we covered the case where one angle is zero we can assume
that only one angle is a right angle).

** TODO: complete.

\begin{thebibliography}{9}

\bibitem{beardon}
  Alan F. Beardon
  The Geometry
  of Discrete Groups
  With 93 Illustrations

\end{thebibliography}

\end{document}



