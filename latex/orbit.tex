\documentclass[a4paper,10pt]{article}
\title{}
\author{}
\usepackage{graphicx} 
\parskip 7.2pt
%% Missing trig. and hyp. operators
\newcommand{\cosec}{\ensuremath{\mathrm{cosec}}}
\newcommand{\arccosec}{\ensuremath{\mathrm{arccosec}}}
\newcommand{\arcsec}{\ensuremath{\mathrm{arcsec}}}
\newcommand{\arccot}{\ensuremath{\mathrm{arccot}}}
\newcommand{\arcsinh}{\ensuremath{\mathrm{arcsinh}}}
\newcommand{\arccosh}{\ensuremath{\mathrm{arccosh}}}
\newcommand{\arctanh}{\ensuremath{\mathrm{arctanh}}}
\newcommand{\arccoth}{\ensuremath{\mathrm{arccoth}}}
\begin{document}
\maketitle
\begin{abstract}

We consider the single parametered family of triangles in the hyperbolic two 
dimensional space that have two angles of size zero and one non zero angle.
More specifically we are interested in the Fangano triangles formed by connecting 
the bases of the (hyperbolic) heights of the triangles as shown in Figure 1, 
modeled as will be the whole document in the upper half plane model:


Our goal is to understand what $\alpha$ maximizes the perimiter of such triangles.
We will see that the perimiter of such inner triangles as a function of the free
angle will of the outer triangle has only one critial point within
the interval $\left[0,\frac{pi}{2}\right)$ located at zero and is thus monotoneus
with respect to $\alpha$ and can easily shown to be decreasing. We will conclude
the maximal perimiter of the Fangano triangle is given in the equilateral
triangle with angles 0-0-0.

In the appendices we will prove that by connecting the bases of the heights of
such triangles we introduce an internal billiard and that that its orbit is indeed
the Fnagano orbit for those triangles.

\end{abstract}
\section{Fangano Orbit length}

We will now show that for acute angle triangles, the length of the fangano orbit
($o$) satisfies:

\begin{center}
(1) $o = 2\arcsinh{\frac{1+2\cosh{a}\cosh{b}\cosh{c}-\cosh{a}^2-\cosh{b}^2-\cosh{c}^2}{\sinh{a}\sinh{b}\sinh{c}}}$
\end{center}

Where $a$, $b$ and $c$ are the lengths of $BC$, $AC$ and $AB$ respectfully and
$\alpha$,$\beta$ and $\gamma$ the angles infront of them.


Proof
Define $A'$, $B'$ and $C'$ as the foots of the heights leaving $A$, $B$ and $C$
respectfully. Define $P$ as the reflecion of $A'$ about $AC$ and $Q$ as the
reflection of $A'$ about $AB$. Since the trigangle $A'B'C'$ is a billiard, it is
easy to see that is perimiter is equal to $d\left(A' Q\right)$.


We further notice that the abgle $PAQ$ is twice the angle $BAC$ and that
the distance from $A$ to $P$ and to $Q$ is equal to thae distance $h$ between
$A$ and $A'$

Recall the cosine rule for triangles $PAQ$ and $ABC$:

\begin{center}
$\frac{\cosh{h}^2-\cosh{x}}{\sinh{h}^2}=\cos{2\alpha}=2\cos{\alpha}^2-1=$

$2\left(\frac{\cosh{b}\cosh{c}-\cosh{a}}{\sinh{b}\sinh{c}}\right)^2 - 1$

$\downarrow$

$\cosh{x} = \cosh^2{h} + \sinh^2{h} - 2\sinh^2{h}\left(\frac{\cosh{b}\cosh{c}-\cosh{a}}{\sinh{b}\sinh{c}}\right)^2 =$

$2 + 2\sinh^2{h} - 2\sinh^2{h}\left(\frac{\cosh{b}\cosh{c}-\cosh{a}}{\sinh{b}\sinh{c}}\right)^2 =$

$2 + 2\sinh^2{h}\left(1 - \left(\frac{\cosh{b}\cosh{c}-\cosh{a}}{\sinh{b}\sinh{c}}\right)^2\right)  $

\end{center}

and, since $BAA'$ is a right angled triangle: 

\begin{center}
$\sinh{h}=\sinh{b}\sin{\beta}$

$\downarrow$

$\sinh^2{h}=\sinh^2{b}\sin^2{\beta}$


\end{center}

Substituting for $h$ :

$\cosh{x} = 2 + 2\sinh^2{b}\sin^2{\beta}\left(1 - \left(\frac{\cosh{b}\cosh{c}-\cosh{a}}{\sinh{b}\sinh{c}}\right)^2\right)$








Let us calculate that distance

\begin{center}
 \includegraphics[width=6cm]{./disk3.pdf}
 % hyper00a.eps: 0x0 pixel, 300dpi, 0.00x0.00 cm, bb=41 259 525 696

 Figure 1
\end{center}




\end{document}



