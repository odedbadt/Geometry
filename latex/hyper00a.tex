\documentclass[a4paper,10pt]{article}
\title{}
\author{}
\usepackage{graphicx} 
\parskip 7.2pt

\begin{document}
\maketitle
\begin{abstract}

We consider the single parametered family of triangles in the hyperbolic two 
dimensional space that have two angles of size zero and one non zero angle.
More specifically we are interested in the Fangano triangles formed by connecting 
the bases of the (hyperbolic) heights of the triangles as shown in Figure 1, 
modeled as will be the whole document in the upper half plane model:

\begin{center}
 \includegraphics[width=6cm]{./sketch00a.pdf}
 % hyper00a.eps: 0x0 pixel, 300dpi, 0.00x0.00 cm, bb=41 259 525 696

 Figure 1
\end{center}

Our goal is to understand what $\alpha$ maximizes the perimiter of such triangles.
We will see that the perimiter of such inner triangles as a function of the free
angle will of the outer triangle has only one critial point within
the interval $\left[0,\frac{pi}{2}\right)$ located at zero and is thus monotoneus
with respect to $\alpha$ and can easily shown to be decreasing. We will conclude
the maximal perimiter of the Fangano triangle is given in the equilateral
triangle with angles 0-0-0.

In the appendices we will prove that by connecting the bases of the heights of
such triangles we introduce an internal billiard and that that its orbit is indeed
the Fnagano orbit for those triangles.

\end{abstract}
\section{Shortest Fangano Triangle}
Let us first find $A'$, $B'$ and $C'$, the bases of the heights. First, it is
trivial to see that:

\begin{center}
$B' = 1 + i$ 
\end{center}

Since the only perpendicular to the euclidean circle that represents the
edge $AC$ that originates in $\infty$ intersects the circle at its highest
point.

Now, since the edges leaving $A$ and $C$ towards $B$ (located at $\infty$) are
perpendicular to the real axis, every hyperbolic line perpendicular to them
is positioned along an Euclidean circle centered at the edge's meeting with the
real axis.

Knowing the above, it is easy to calculate the point the perpendiculars cross
the opposie edges. Indeed, we have shown the center is located at the edge's
meeting point with the real axis and since it passes through a known vertex, 
the radius is easilt calculated as the distance bewteem the center and the
vertex.

\begin{center}
$A' = i \sqrt{2 + 2 \cos{\alpha}}$

$C' = \left(1 + i\right) \left(1 + \cos{\alpha}\right)$
\end{center}

Now let us calculate the hyperbolic distances between $a$, $b$ and $c$ using the formula:

\begin{center}
$d\left(x,y\right) = acosh(\frac{\|x-y\|^{2}}{\Im\left(x\right)\Im\left(y\right)})$
\end{center}

\noindent$d\left(A', B'\right) =$

$acosh\left(1 + \frac{\left(\Im\left(A' - B'\right)\right)^{2} + \left(\Re\left(A' - B'\right)\right)^{2}}{2 \sqrt{2 + 2 \cos{\alpha}}}\right) =$
$acosh\left(1 + \frac{1 + \left(1 - \sqrt{2 + 2 \cos{\alpha}}\right)^{2}}{2 \sqrt{2 + 2 \cos{\alpha}}}\right) =$

$acosh\left(1 + \frac{4 + 2 \cos{\alpha} - 2 \sqrt{2 + 2 \cos{\alpha}}}{2 \sqrt{2 + 2 \cos{\alpha}}}\right) =$
$acosh\left(\frac{4 + 2 \cos{\alpha}}{2 \sqrt{2 + 2 \cos{\alpha}}}\right) =$

$acosh\left(\frac{2 + \cos{\alpha}}{\sqrt{2 + 2 \cos{\alpha}}}\right) $

\noindent$d\left(B', C'\right) = $

$acosh\left(1 + \frac{\left(\Im\left(B' - B'\right)\right)^{2} + \left(\Re\left(B' - C'\right)\right)^{2}}{2 + 2 \cos{\alpha}}\right) =$
$acosh\left(1 + 2 \frac{cos^{2}{\alpha}}{2 + 2 \cos{\alpha}}\right) =$

$acosh\left(1 + \frac{cos^{2}{\alpha}}{1 + \cos{\alpha}}\right) =$
$acosh\left(\frac{1 + \cos{\alpha} + cos^{2}{\alpha}}{1 + \cos{\alpha}}\right) $

\noindent$ d\left(C', A'\right)=$

$acosh\left(1 + \frac{\left(\Im\left(C' - A'\right)\right)^{2} + \left(\Re\left(C' - A'\right)\right)^{2}}{\left(2 + 2 \cos{\alpha}\right)^{\frac{3}{2}}}\right) =$

$acosh\left(1 + \frac{\left(1 + \cos{\alpha}\right)^{2} + \left(1 + \cos{\alpha} - \sqrt{2 + 2 \cos{\alpha}}\right)^{2}}{\left(2 + 2 \cos{\alpha}\right)^{\frac{3}{2}}}\right) =$

$acosh\left(1 + \frac{2\left(1 + \cos{\alpha}\right)^{2} + 2 + 2 \cos{\alpha} - 2\left(1 + \cos{\alpha}\right)\sqrt{2 + 2 \cos{\alpha}}}{\left(2 + 2 \cos{\alpha}\right)^{\frac{3}{2}}}\right) =$

$acosh\left(1 + \frac{2 + 2cos^{2}{\alpha} + 2\cos{\alpha} + 2 + 2 \cos{\alpha} - 2\left(1 + \cos{\alpha}\right)\sqrt{2 + 2 \cos{\alpha}}}{\left(2 + 2 \cos{\alpha}\right)^{\frac{3}{2}}}\right) =$

$acosh\left(1 + \frac{4 + 6 \cos{\alpha} + 2 cos^{2}{\alpha} - \left(2 + 2 \cos{\alpha}\right)^{\frac{3}{2}}}{\left(2 + 2 \cos{\alpha}\right)^{\frac{3}{2}}}\right) =$

$acosh\left(\frac{\left(2 + 2 \cos{\alpha}\right)^{\frac{3}{2}} + 4 + 6 \cos{\alpha} + 2 cos^{2}{\alpha} - \left(2 + 2 \cos{\alpha}\right)^{\frac{3}{2}}}{\left(2 + 2 \cos{\alpha}\right)^{\frac{3}{2}}}\right) =$

$acosh\left(\frac{4 + 6 \cos{\alpha} + 2 cos^{2}{\alpha}}{\left(2 + 2 \cos{\alpha}\right)^{\frac{3}{2}}}\right) =$

$acosh\left(\frac{2 \left(2 + 3 \cos{\alpha} + cos^{2}{\alpha}\right)}{\left(2 + 2\cos{\alpha}\right)^{\frac{3}{2}}}\right) =$

$acosh\left(\frac{2 \left(1 + \cos{\alpha}\right)\left(2 + \cos{\alpha}\right)}{\left(2 + 2\cos{\alpha}\right) \sqrt{\left(2 + 2\cos{\alpha}\right) } } \right) =$

$acosh\left(\frac{2 + \cos{\alpha}}{\sqrt{2 + 2 \cos{\alpha}}}\right)$

\noindent
We notice that
\begin {center}
$ d\left(C', A'\right)=d\left(A', B'\right) $
\end {center}

\noindent
We differentiate each distance expression (with respect to $\alpha$):

\noindent$\frac{d}{d\alpha}d\left(A', B'\right)=\frac{d}{d\alpha}d\left(C', A'\right)=$

$ \left(\left(-1 + \left( \frac{2 + \cos{\alpha}}
                        {\sqrt{2 + 2 \cos{\alpha}}}\right ) ^{2}\right)\right)^{-\frac{1}{2}}    \frac{d}{d\alpha}\left (\frac{2 + \cos{\alpha}}{\sqrt{2 + 2 \cos{\alpha}}}\right) =$

$ \left(\left(-1 + \frac{\left(2 + \cos{\alpha}\right)^{2}}{2 + 2 \cos{\alpha}}\right)\right)^{-\frac{1}{2}}  \frac{1}{\sqrt{2}}\frac{d}{d\alpha}\left (\sqrt{1 + \cos{\alpha}} + \frac{1}{\sqrt{1 + \cos{\alpha}}}\right) =$

$ \frac{1}{\sqrt{2}} \left(\frac{4 + cos^{2}{\alpha} + 4\cos{\alpha} - 2 - 2\cos{\alpha}}{2 + 2\cos{\alpha}}\right)^{-\frac{1}{2}}
  \left(\frac{1}{2}\left(1 + \cos{\alpha}\right)^{-\frac{1}{2}} -\frac{1}{2}\left(1 + \cos{\alpha}\right)^{-\frac{3}{2}}\right)\left(-\sin{\alpha}\right) =$

$ \frac{1}{\sqrt{2}}\sqrt{2}\left(2 + cos^{2}{\alpha} + 2\cos{\alpha}\right)^{-\frac{1}{2}} \left(1 + \cos{\alpha}\right)^{\frac{1}{2}}
  \left(\frac{1}{2}\left(1 + \cos{\alpha}\right)^{-\frac{1}{2}} -\frac{1}{2}\left(1 + \cos{\alpha}\right)^{-\frac{3}{2}}\right)\left(-\sin{\alpha}\right) =$

$ \left(2 + cos^{2}{\alpha} + 2\cos{\alpha}\right)^{-\frac{1}{2}} 
  \frac{1}{2}\left(\left(1 + \cos{\alpha}\right)^{0} -\left(1 + \cos{\alpha}\right)^{-1}\right)\left(-\sin{\alpha}\right) =$

$ \frac{1}{\sqrt{  2 + cos^{2}{\alpha} + 2\cos{\alpha}  }}
  \frac{1}{2}\left(1 - \frac{1}{1 + \cos{\alpha}}\right)\left(-\sin{\alpha}\right) =$

$ \frac{-\sin{\alpha}}{2\sqrt{ cos^{2}{\alpha} + 2\cos{\alpha} + 2 }}
  \frac{\cos{\alpha}}{1 + \cos{\alpha}} =$

$ -\frac{\sin{\alpha}\cos{\alpha}}{2\left(1 + \cos{\alpha}\right)\sqrt{ cos^{2}{\alpha} + 2\cos{\alpha} + 2 }}$

\noindent$\frac{d}{d\alpha}d\left(B',C'\right)=$

$ \left(\left(-1 + \left( \frac{1 + \cos{\alpha} + cos^{2}{\alpha}}{1 + \cos{\alpha}}
                   \right ) ^{2}\right)\right)^{-\frac{1}{2}}
   \frac{d}{d\alpha} \left (\frac{1 + \cos{\alpha} + cos^{2}{\alpha}}{1 + \cos{\alpha}}
                           \right) =$



$ \left(\left(-1 + \left(1 +  \frac{cos^{2}{\alpha}}{1 + \cos{\alpha}}
                   \right ) ^{2}\right)\right)^{-\frac{1}{2}}
   \frac{d}{d\alpha} \left (1 + \frac{cos^{2}{\alpha}}{1 + \cos{\alpha}}
                           \right) =$


$ -\left(\left( 2\frac{cos^{2}{\alpha}}{1 + \cos{\alpha}} + \left(\frac{cos^{2}{\alpha}}{1 + \cos{\alpha}} \right ) ^{2}
        \right)
   \right)^{-\frac{1}{2}}
  \frac{2\cos{\alpha}\left(1+\cos{\alpha}\right) - cos^2{\alpha}}{\left(1 + \cos{\alpha}\right)^2}\sin{\alpha}$


$ -\sin{\alpha}\left(  
  \frac{ 2cos^{2}{\alpha}\left(1 + \cos{\alpha}\right) + cos^4{\alpha} }
       {\left(1 + \cos{\alpha}\right)^2}
   \right)^{-\frac{1}{2}}
  \frac{2\cos{\alpha} + cos^2{\alpha}}{\left(1 + \cos{\alpha}\right)^2}$

$ -\frac{\sin{\alpha}\left(1 + \cos{\alpha}\right)\cos{\alpha}\left(2  +\cos{\alpha}\right)}
   {\cos{\alpha}\sqrt{ 2\left(1 + \cos{\alpha}\right) + cos^2{\alpha} }\left(1 + \cos{\alpha}\right)^2}  =$


$- \frac{\left(2 + \cos{\alpha}\right) \sin{\alpha}}{\sqrt{2 + 2 \cos{\alpha} + cos^{2}{\alpha}} \left(1 + \cos{\alpha}\right)}$

\noindent
And sum the three up:

\noindent$\frac{d}{d\alpha}\left(d\left(A', B'\right) + d\left(B', C'\right) + d\left(C', A'\right)\right) = $

$\frac{d}{d\alpha}\left(2d\left(A', B'\right) + d\left(B', C'\right)\right) = $

$2\frac{d}{d\alpha}d\left(A', B'\right) + \frac{d}{d\alpha}d\left(B', C'\right) = 
 - \frac{\left(2 + \cos{\alpha}\right) \sin{\alpha}}{\sqrt{2 + 2 \cos{\alpha} + cos^{2}{\alpha}} \left(1 + \cos{\alpha}\right)} - $
  
$  2\frac{\cos{\alpha} \sin{\alpha}}
  {\sqrt{2 + 2 \cos{\alpha} + cos^{2}{\alpha}} \left(2 + 2 \cos{\alpha}\right)} =$

$  \frac{-\left(2 + \cos{\alpha}\right) \sin{\alpha} - \cos{\alpha} \sin{\alpha}}{\sqrt{2 + 2 \cos{\alpha} + cos^{2}{\alpha}} \left(1 + \cos{\alpha}\right)} $

$- \frac{2\sin{\alpha}\left(1 + \cos{\alpha}\right)}{\sqrt{2 + 2 \cos{\alpha} + cos^{2}{\alpha}\left(1 + \cos{\alpha}\right)}} =$
$- \frac{2\sin{\alpha}}{\sqrt{2 + 2 \cos{\alpha} + cos^{2}{\alpha}}}$

\noindent
We end up with an expression that is negative and null only at zero (within the
interval $\left[0,frac{\pi}{2}\right)$) and can conclude that the maximal perimiter
of the Fangano triangle is achieved at $\alpha = 0$. Its perimiter is:

\begin{center}
$2acosh\left(\frac{2 + cos\left(0\right)}{\sqrt{2 + 2 cos\left(0\right)}}\right) + 
acosh\left(\frac{1 + cos\left(0\right) + cos^{2}\left(0\right)}{1 + cos\left(0\right)}\right) =
3acosh\left(\frac{3}{2}\right)$
\end{center}


\section{Appendix A}
We will provide a short proof of the fact that the triangle connecting the bases
of the heights of such triangles (with two null angles) is indeed a billiard orbit.
 
We will achieve the above by finding the equations of the edges and showing the
angles of intersection with the edges are equal to each other, which is easy to
see means the deriviative's are equal in absolute values but opposite in sign.

First we calculate the centers of the Euclidean circles along which the edges
of the inner triangle $A'B'C'$ lay using the formula:

\begin{center}
  $c\left(x,y\right) = \frac{\left|x\right|^{2} - \left|y\right|^{2}}{2\left(\Re(x) - \Re(y)\right)} $
\end{center}

Where $c\left(x,y\right)$ represents the center of the Euclidean circle passing
through $x$ and $y$.

  $c\left(A',B'\right) = \frac{\left(2 + 2\cos{\alpha} - 2\right)}{{2\left(0 - 1\right)}} = -cos(\alpha) $

  $c\left(B',C'\right) = \frac{2 - 2\left(1 + \cos{\alpha}\right)^2}{2\left(1 - 1 - \cos{\alpha}\right)} = $
  $ \frac{1 - \left(1 + \cos{\alpha}\right)^2}{-\cos{\alpha}} = \cos{\alpha} + 2$

  $ \frac{-2\cos{\alpha} - cos^{2}{\alpha}}{-\cos{\alpha}} = \cos{\alpha} + 2$

  $c\left(C',A'\right) = \frac{2\left(1 + \cos{\alpha} \right)^2 - \left(2 + 2\cos{\alpha}\right)}{2\left(1+\cos{\alpha} - 0\right)} =$

  $\frac{\left(1 + \cos{\alpha} \right)^2 - \left(1 + \cos{\alpha}\right)}{1+\cos{\alpha}} =$
  $\frac{\left(1 + \cos{\alpha} \right)^2 - \left(1 + \cos{\alpha}\right)}{1+\cos{\alpha}} = 1 + \cos{\alpha} - 1 = $
  $\cos{\alpha}$

It is easy to show hat the differential equation describing a circle whose center
$r$ is on the real axis can be written as:

$2\left(x-r\right)+2yy'=0$

Sustitutibg the above calculated centers:
  
\noindent $AB$ :  $y' = -\frac{x + \cos{\alpha}}{ y} $

\noindent $BC$ :  $y' = -\frac{x - cos\left(a\right) - 2}{y}$

\noindent $CA$ :  $y' = -\frac{x - cos\left(a\right)}{y}$


Substituting $(x,y)$ for the real and imaginary parts of the endpoints $A'$,$B'$,$C'$ we get:

\noindent $A$: 
on $CA$ : $-\frac{-\cos{\alpha}}{\sqrt{2 + 2 \cos{\alpha}}} = \frac{\cos{\alpha}}{\sqrt{2 + 2 \cos{\alpha}}}$

on $AB$ : $-\frac{\cos{\alpha}}{\sqrt{2 + 2 \cos{\alpha}}}$

\noindent $B$: 
on $AB$ : $-\left(1 + \cos{\alpha}\right)$

on $BC$ : $-\left(1 - \cos{\alpha} - 2\right)$ =  $1 + \cos{\alpha}$

\noindent $C$: 
on $BC$ : $-\frac{1 + \cos{\alpha} - cos\left(a\right) - 2}{1 + \cos{\alpha}} = \frac{1}{1 + \cos{\alpha}}$

on $CA$ : $-\frac{1 + \cos{\alpha} - cos\left(a\right)    }{1 + \cos{\alpha}} = -\frac{1}{1 + \cos{\alpha}}$


\end{document}



